\documentclass[11pt,]{article}
\usepackage[left=1in,top=1in,right=1in,bottom=1in]{geometry}
\newcommand*{\authorfont}{\fontfamily{phv}\selectfont}
\usepackage[]{mathpazo}


  \usepackage[T1]{fontenc}
  \usepackage[utf8]{inputenc}



\usepackage{abstract}
\renewcommand{\abstractname}{}    % clear the title
\renewcommand{\absnamepos}{empty} % originally center

\renewenvironment{abstract}
 {{%
    \setlength{\leftmargin}{0mm}
    \setlength{\rightmargin}{\leftmargin}%
  }%
  \relax}
 {\endlist}

\makeatletter
\def\@maketitle{%
  \newpage
%  \null
%  \vskip 2em%
%  \begin{center}%
  \let \footnote \thanks
    {\fontsize{18}{20}\selectfont\raggedright  \setlength{\parindent}{0pt} \@title \par}%
}
%\fi
\makeatother




\setcounter{secnumdepth}{0}



\title{What Should \emph{Financial} Economists Do?  }



\author{\Large Tyler J. Brough\vspace{0.05in} \newline\normalsize\emph{Utah State University}  }


\date{}

\usepackage{titlesec}

\titleformat*{\section}{\normalsize\bfseries}
\titleformat*{\subsection}{\normalsize\itshape}
\titleformat*{\subsubsection}{\normalsize\itshape}
\titleformat*{\paragraph}{\normalsize\itshape}
\titleformat*{\subparagraph}{\normalsize\itshape}


\usepackage{natbib}
\bibliographystyle{apsr}



\newtheorem{hypothesis}{Hypothesis}
\usepackage{setspace}

\makeatletter
\@ifpackageloaded{hyperref}{}{%
\ifxetex
  \usepackage[setpagesize=false, % page size defined by xetex
              unicode=false, % unicode breaks when used with xetex
              xetex]{hyperref}
\else
  \usepackage[unicode=true]{hyperref}
\fi
}
\@ifpackageloaded{color}{
    \PassOptionsToPackage{usenames,dvipsnames}{color}
}{%
    \usepackage[usenames,dvipsnames]{color}
}
\makeatother
\hypersetup{breaklinks=true,
            bookmarks=true,
            pdfauthor={Tyler J. Brough (Utah State University)},
             pdfkeywords = {Austrian school, financial economics, catallactics},  
            pdftitle={What Should \emph{Financial} Economists Do?},
            colorlinks=true,
            citecolor=blue,
            urlcolor=blue,
            linkcolor=magenta,
            pdfborder={0 0 0}}
\urlstyle{same}  % don't use monospace font for urls



\begin{document}
	
% \pagenumbering{arabic}% resets `page` counter to 1 
%
% \maketitle

{% \usefont{T1}{pnc}{m}{n}
\setlength{\parindent}{0pt}
\thispagestyle{plain}
{\fontsize{18}{20}\selectfont\raggedright 
\maketitle  % title \par  

}

{
   \vskip 13.5pt\relax \normalsize\fontsize{11}{12} 
\textbf{\authorfont Tyler J. Brough} \hskip 15pt \emph{\small Utah State University}   

}

}







\begin{abstract}

    \hbox{\vrule height .2pt width 39.14pc}

    \vskip 8.5pt % \small 

\noindent In this essay I argue for a stronger link between the Austrian School
and financial economics.


\vskip 8.5pt \noindent \emph{Keywords}: Austrian school, financial economics, catallactics \par

    \hbox{\vrule height .2pt width 39.14pc}



\end{abstract}


\vskip 6.5pt

\noindent \doublespacing \section{Introduction}\label{introduction}

In his profound essay \citet{Buchanan1964} asks: ``What Should
Economists Do?''. In this essay, I ask a similar question but direct my
remarks to the field of finance. Here, I ask: ``What Should
\textbf{\emph{Financial Economists}} Do?'' My answer, following
Buchanan, is that financial economists should focus on economic exchange
rather than on the narrow topic of economic optimization. I suggest that
financial economists ought to be very comfortable with this paradigm.
Indeed, the field of market microstructure finance really is a
\textbf{\emph{catallactics}} of financial markets, though it does not go
by that name. Finance is a heavily empirical discipline, and as
\citet{Culp2004} has said, ``it {[}is{]} quite hard to explain a lot of
financial and derivatives market activity without being a \emph{little
bit} Austrian.'' My contention, is that financial economists should
actively embrace an Austrian foundation for the study of their
discipline, and that if they do they will see the questions they face
with greater clarity.

Only rarely are works in the Austrian tradition cited in the modern
finance literature. This despite the fact that much of the modern theory
of finance is founded on the ideas of informationally efficient prices
following the foundational ideas of \citet{Hayek1945}, such as the
entire Efficient Markets Hypothesis literature. For example, neither of
the papers by \citet{Samuelson1965} and \citet{Fama1970} cited
\citet{Hayek1945}. Had they done so, I argue that greater clarity would
have accompanied the following development of the modern finance
literature. Thankfully, there are some rare exceptions such as
\citet{Grossman1989} and \citet{Vives2010}.

Hayek has pointed out that arbitrage is the central concept in all of
economics. See for example the discussion between Buchanan and Hayek
\citep{BuchananHayek} and the discussion by Israel Kirzner
\citep{KirznerYT1}. While arbitrage is central to all of economics,
nowhere in the broad economic literature has the concept been more
developed than in finance. In the finance literature arbitrage has taken
on a very specific definition requiring riskless profit (e.g.
\citep{Varian1987} or \citep{DybvigRoss1989}). This has lead to a lack
of coherence between theoretical finance and the vast literature on
empirical finance. For example, the famed model of option pricing by
\citet{BlackScholes1973} requires that there be no arbitrage
opportunities between the option and the dynamic hedge portfolio at
every infinitesimal point in time. This extreme form of equilibrium
theory is of course impossible, and taken literally is absurd. This has
been implicitly understood by empirical researchers in finance. Still,
young scholars in finance are often confused when moving from their
theoretical asset pricing and empirical asset pricing courses. This is
because theoretical finance is strongly grounded in neoclassical
equilibrium theory, while empirical researchers have per force had to
deal with real-world market dynamics evident in their data. I argue that
a deeper understanding of Austrian market process theory would go a long
way to harmonizing our understanding of theory and empirics. It will
also help to focus our empirical lens by informing us on the nature of
the questions that are amenable to econometric modeling.

At the same time, many powerful statistical and econometric tools have
been developed in the empirical finance literature that ought to be
informative to the rest of economics if indeed the arbitrage principle
is central to economics as Hayek thought. A powerful example of this is
the work on cointegration and error correction models due to
\citet{EngleGranger1987}. This model has been given an explicit Austrian
interpretation by \citet{Mulligan2005}. As \citet{Koop2006} says:
``{[}I{]}t can be argued that cointegration is how macroeconomic
equilibrium concepts should manifest themselves empirically.'' As
Mulligan points out without an understanding of market process theory it
becomes very challenging to understand the results of such a model. For
a typical use of cointegration in finance see \citet{Liu2005}. Liu finds
a cointegrating relationship between hog, corn and soybean futures
prices, suggesting a long-run (no-arbitrage) equilibrium. At the same he
finds that ``ex-post trading simulations that utilize the cointegration
results generate significant profits, suggesting that market
expectations may not fully incorporate the mean-reverting tendencies as
indicated by the cointegration relations, and that inefficiency exists
in these three commodity futures markets.'' From the perspective of
neoclassical finance, this is indeed puzzling. But from an Austrian
market process perspective this is entirely expected. From this latter
perspective markets are dynamically efficient and must per force allow
for arbitrage possibilities in the short-run. The empirical finance
literature is chock-full of such findings. I argue that other fields in
economics can learn from the many empirical successes in empirical
finance. Perhaps the most fruitful application of this kind of empirical
arbitrage-measurement thinking will be in the field of industrial
organization. Indeed, researchers such as \citet{Spulber1999} have
pointed out the connection between industrial organization and market
microstrucutre finance. By giving microstructure an explicit Austrian
foundation many issues will be clarified.

Despite the many empirical successes in finance there remains skepticism
regarding the role of econometrics. See for example, \citet{Black1982},
who is deeply skeptical of econometric methods as traditionally
conceived. Here too the Austrian tradition informs of us how to
understand these issues. \citet{Hayek1994} wrote that our only knowledge
of economic laws, was infact knowledge of a pattern, and that pattern
prediction was the main methodological avenue of discovery. When viewed
from this angle econometrics begins to look more like its modern cousins
in the statistical sciences of statistical signal processing and machine
learning. Financial market participants have discovered this as well. A
inspection of job posting in quantitative finance rarely require skills
in econometrics, while the latter are often cited as required. I argue
that the best and most successful examples of financial econometrics can
most profitable be understood from this angle. For example, the vast
literature on cointegration in finance can be seen as prediction of
pattern. Indeed, the central pattern in all of economics: arbitrage and
the dynamic tendecy towards a single equilibrium price. In this light
the examples of financial econometrics can be successfully applied in
other fields in economics.

In summary, in this essay I will seek to establish more concretely the
linkage between modern financial economics and the Austrian school of
thought. I will highlight the many ways in which finance has essentially
re-discovered the foundational concepts of the informational role of
prices due to \citet{Hayek1945}, and the central role of arbitrage in
market process theory.

\newpage
\singlespacing 
\bibliography{master.bib}

\end{document}
