\documentclass[11pt,]{article}
\usepackage[left=1in,top=1in,right=1in,bottom=1in]{geometry}
\newcommand*{\authorfont}{\fontfamily{phv}\selectfont}
\usepackage[]{mathpazo}


  \usepackage[T1]{fontenc}
  \usepackage[utf8]{inputenc}



\usepackage{abstract}
\renewcommand{\abstractname}{}    % clear the title
\renewcommand{\absnamepos}{empty} % originally center

\renewenvironment{abstract}
 {{%
    \setlength{\leftmargin}{0mm}
    \setlength{\rightmargin}{\leftmargin}%
  }%
  \relax}
 {\endlist}

\makeatletter
\def\@maketitle{%
  \newpage
%  \null
%  \vskip 2em%
%  \begin{center}%
  \let \footnote \thanks
    {\fontsize{18}{20}\selectfont\raggedright  \setlength{\parindent}{0pt} \@title \par}%
}
%\fi
\makeatother




\setcounter{secnumdepth}{0}



\title{The Political Economy of Nonlinear Dynamics in Dairy Prices  }



\author{\Large Tyler J. Brough\vspace{0.05in} \newline\normalsize\emph{Utah State University}  }


\date{}

\usepackage{titlesec}

\titleformat*{\section}{\normalsize\bfseries}
\titleformat*{\subsection}{\normalsize\itshape}
\titleformat*{\subsubsection}{\normalsize\itshape}
\titleformat*{\paragraph}{\normalsize\itshape}
\titleformat*{\subparagraph}{\normalsize\itshape}


\usepackage{natbib}
\bibliographystyle{apsr}



\newtheorem{hypothesis}{Hypothesis}
\usepackage{setspace}

\makeatletter
\@ifpackageloaded{hyperref}{}{%
\ifxetex
  \usepackage[setpagesize=false, % page size defined by xetex
              unicode=false, % unicode breaks when used with xetex
              xetex]{hyperref}
\else
  \usepackage[unicode=true]{hyperref}
\fi
}
\@ifpackageloaded{color}{
    \PassOptionsToPackage{usenames,dvipsnames}{color}
}{%
    \usepackage[usenames,dvipsnames]{color}
}
\makeatother
\hypersetup{breaklinks=true,
            bookmarks=true,
            pdfauthor={Tyler J. Brough (Utah State University)},
             pdfkeywords = {derivatives markets, dairy markets, nonlinear dynamics, price discovery,
signal processing},  
            pdftitle={The Political Economy of Nonlinear Dynamics in Dairy Prices},
            colorlinks=true,
            citecolor=blue,
            urlcolor=blue,
            linkcolor=magenta,
            pdfborder={0 0 0}}
\urlstyle{same}  % don't use monospace font for urls


\def\tightlist{}

\begin{document}
	
% \pagenumbering{arabic}% resets `page` counter to 1 
%
% \maketitle

{% \usefont{T1}{pnc}{m}{n}
\setlength{\parindent}{0pt}
\thispagestyle{plain}
{\fontsize{18}{20}\selectfont\raggedright 
\maketitle  % title \par  

}

{
   \vskip 13.5pt\relax \normalsize\fontsize{11}{12} 
\textbf{\authorfont Tyler J. Brough} \hskip 15pt \emph{\small Utah State University}   

}

}







\begin{abstract}

    \hbox{\vrule height .2pt width 39.14pc}

    \vskip 8.5pt % \small 

\noindent In this project I investigate the nonlinear dynamics of dairy spot and
derivative markets prices. I relate these dynamics to policies by
regulators.


\vskip 8.5pt \noindent \emph{Keywords}: derivatives markets, dairy markets, nonlinear dynamics, price discovery,
signal processing \par

    \hbox{\vrule height .2pt width 39.14pc}



\end{abstract}


\vskip 6.5pt

\noindent \doublespacing \section{Introduction}\label{introduction}

In this project, I examine the many observed empirical facts of dairy
market spot, futures, and options prices and relate them to the various
dairy market policies that have been put in place over the years to
regulate dairy products. I argue that although the intended consequences
of the policies are to stabilize dairy market prices, these policies
have had the exact opposite effect and have resulted in highly volatile
prices with many observable nonliear dynamic features.

This project is still very nascent. This proposal is a barebones outline
of the project as currently conceived.

\subsection{Empirical Features of Dairy Market
Prices}\label{empirical-features-of-dairy-market-prices}

Dairy spot and futures prices exhibit many interesting features from the
perspective of time series econometrics and financial theory:

\begin{itemize}
\tightlist
\item
  Dairy prices exhibit extreme volatility
\item
  Dairy prices exhibit time-varying volatility and other parameters
\item
  Dairy prices exhibit many structural breaks
\item
  Dairy prices exhibit extreme non-Gaussianity
\end{itemize}

In sum, there are many nonlinear and complex dynamic features in dairy
spot and derivatives prices. I will seek to demonstrate that government
regulatory policies have been anything other than stabilizing. From the
perspective of the informational role of prices due to
\citet{Hayek1945}, this unsurprising. Regulators do not possess all of
the vast information that would be required to actually successfully
regulate and stabalize prices. Much of the complex dynamics exhibited in
dairy market prices are likely due to market participants seeking to
process their local information in the presence of the noise added by
government regulations.

\subsection{Dairy Market Price Data}\label{dairy-market-price-data}

I am now collecting data from the Commodity Research Bureau (CRB) for
all historical dairy spot, futures and options markets. This will result
in a quite massive dataset. Efficient data processing techniques are
currently being investigated. See \citet{WuBethelGuLeinweberRubel} for
example.

\subsection{Empirical Methods}\label{empirical-methods}

The empirical methods that will be employed in this study include the
following:

\begin{itemize}
\tightlist
\item
  The Bayesian prior predictive and posterior predictive methods of
  \citet{GewekeBook2} and \citet{LancasterBook}.
\item
  The Markov-switching Bayesian vector error-correction model of
  \citet{JochmannKoop2015}.
\item
  The Multivariate Singular Spectrum Analysis due to
  \citet{HassaniThomakos2010}.
\end{itemize}

The objective of the empirical analysis is to decompose the time series
signal into its various components, and to estimate the structural break
points. My prior is that they will line up well with policy changes.

\subsection{Summary}\label{summary}

I believe that I will be able to get at least two published articles
from this effort. The first will focus on the econometrics of signal
processing of the dairy price series. The second will focus on the
political economy of dairy market regulations, relying on the empirical
findings of the first paper.

The project has deep connections to what I call the Hayekian approach to
finance that is outlined in my other proposals.

\newpage
\singlespacing 
\bibliography{master.bib}

\end{document}
